\documentclass{article}
\usepackage{amsmath}
\usepackage{graphicx}
\usepackage{color}
\usepackage[left=1in,right=1in,top=1in,bottom=1in]{geometry}
\usepackage[utf8]{inputenc}
%\usepackage{hyperref}

\usepackage{hyperref}
%\usepackage[hypcap]{caption}
%\usepackage[backend=bibtex,style=nature]{biblatex}
\usepackage[backend=bibtex,style=nature]{biblatex}
\usepackage[english]{babel}
\usepackage{mathrsfs}
\usepackage{amssymb}
\usepackage{latexsym}

\usepackage{url}
\usepackage[nodisplayskipstretch]{setspace}
\setstretch{1}
\usepackage[percent]{overpic}
\usepackage{mdframed}
\usepackage{framed}
\usepackage{bm}
\usepackage{bigints}
%\usepackage{mtpro2}
\usepackage{relsize}
\usepackage{mathtools}
\usepackage{scalerel}
%\usepackage{classicthesis-ldpkg}
\usepackage[titletoc,toc,title]{appendix}
\usepackage{chngcntr}
\usepackage{empheq}
%\usepackage{epstopdf}
\usepackage{rotating}
%\usepackage{cleveref}
\usepackage{array}
\usepackage{tabularx}
\usepackage{caption}
\usepackage{subcaption}
\usepackage{multirow}
\usepackage{calrsfs}

\usepackage{tikz}
\usepackage{colortbl}
\newcommand\y{\cellcolor{clight2}}
\definecolor{clight2}{RGB}{212, 237, 244}%
\newcommand\tikznode[3][]%
{\tikz[remember picture,baseline=(#2.base)]
	\node[minimum size=0pt,inner sep=0pt,#1](#2){#3};%
}
\tikzset{>=stealth}
\renewcommand\vec[1]{\mathbf{#1}}


\parindent = 0pt
%-----------------------------------------------------------------
%-----------------------------------------------------------------

\hypersetup{colorlinks=true,linkcolor=blue,filecolor=blue,urlcolor=blue,citecolor=blue}

%-----------------------------------------------------------------
%-----------------------------------------------------------------

\newcommand*\widefbox[1]{\fbox{\hspace{2em}#1\hspace{2em}}}
\newcommand{\citelink}[2]{\hyperlink{cite.\therefsection @#1}{#2}}
\newcommand{\bDiamond}{\mathbin{\Diamond}}
\newcommand{\bLozenge}{\mathbin{\blacklozenge}}

\newcommand{\lone}{ \bar{\lambda}_{1} }
\newcommand{\ltwo}{ \bar{\lambda}_{2} }
\newcommand{\lthree}{ \bar{\lambda}_{3} }

\newcommand{\li}{ \bar{\lambda}_{i} }
\newcommand{\Rb}{ \bar{R} }
\newcommand{\myJ}{ \mathcal{J}}

\newcommand{\PDfirst}[2]{ \dfrac{\partial #1}{\partial #2}  }

%-----------------------------------------------------------------
%-----------------------------------------------------------------


\def\stretchint#1{\vcenter{\hbox{\stretchto[220]{\displaystyle\oint}{#1}}}}
\def\scaleint#1{\vcenter{\hbox{\scaleto[30ex]{\displaystyle\oint}{#1}}}}

%-----------------------------------------------------------------
%-----------------------------------------------------------------

\addbibresource{bible.bib}

%-----------------------------------------------------------------
%-----------------------------------------------------------------

%\graphicspath{ {C:/Users/r.pourmodheji/Desktop/1st/Images_all/} }
%\graphicspath{ {/Users/Reza/Desktop/2nd/Images_all/} }

%-----------------------------------------------------------------
%-----------------------------------------------------------------
\newcolumntype{S}[2]{>{ \minipage[c][#1][c]{#2} }c<{\raggedright \arraybackslash \endminipage}}
\newcolumntype{R}[2]{>{ \minipage[c][#1][c]{#2} }c<{\centering \arraybackslash \endminipage}}
%#################################################################
%#################################################################
%#################################################################
\begin{document}
	
	\fontsize{12}{14}\selectfont
	
	\setlength{\belowdisplayskip}{20pt} \setlength{\belowdisplayshortskip}{20pt}
	\setlength{\abovedisplayskip}{10pt} \setlength{\abovedisplayshortskip}{10pt}
	
	%#################################################################
%	\begin{center}
%		{\fontsize{20}{22}\selectfont
%			Energy based strength theory for soft membranes \\[1em]}
%		
%		Reza Pourmodheji\textsuperscript{a}~~~~~~~~Shaoxing Qu\textsuperscript{b}~~~~~~~~Honghui Yu\textsuperscript{a*}\\[2ex]
%		
%		{\fontsize{8}{10}\selectfont
%			\textsuperscript{a}Department of Mechanical Engineering
%			The City College of New York, New York, NY 10031\\
%			\textsuperscript{b}State Key Laboratory of Fluid 
%			Power and Mechatronic, 
%			Key Laboratory of Soft Machines and 
%			Smart Devices of Zhejiang Province, 
%			Department of Engineering Mechanics, 
%			Zhejiang University, 
%			Hangzhou 310027, China }
%	\end{center}
%	
%	
	
	
	\section*{Heat Equation}
	
	We are supposed to solve the Initial-Boundary Volume Problem.
	
	
	\begin{gather}
			{\partial u\over\partial t} = \nabla^2 u + f \quad \hbox{in }\Omega\times(0, T], \\
			u = u_{_\mathrm{D}} \hbox{on } \partial \Omega\times(0, T], \\
			u = u_{_0} \mbox{at } t=0{\thinspace .}
	\end{gather}
	
	So we are supposed to find $u(x,y,t)$ in the domain $\Omega$ through $(0,T]$.
	
	
	We first discretize in time $t$. So say
	
	\begin{gather}
	{\partial u\over\partial t} = \dfrac{u^{n+1}(x,y)-u^n(x,y)}{\Delta t}
	\end{gather}
	
	The ${n+1}$ means at the time $t=t_{n+1}$. Thus
	
	\begin{gather}
	 \dfrac{u^{n+1}(x,y)-u^n(x,y)}{\Delta t} = \nabla^2u^{n+1}(x,y)+ f^{n+1}(x,y)
	\end{gather}
	
	and
	
	
	\begin{gather}
		u^{n+1}(x,y)-u^n(x,y) = \nabla^2u^{n+1}(x,y)\Delta t + f^{n+1}(x,y) \Delta t
	\end{gather}
	
	or
	
	\begin{gather}
	u^{n+1}(x,y) - \nabla^2u^{n+1}(x,y)\Delta t  = u^n(x,y) + f^{n+1}(x,y) \Delta t
	\end{gather}
	
	Now take a function $v(x,y,t)$ which vanishes on the boundaries and multiply the PDE and the initial conditions by it so
	
	\begin{gather}
	\big( u^{n+1}(x,y) - \nabla^2u^{n+1}(x,y)\Delta t \big)v^{n+1}(x,y)  = \big(u^n(x,y) + f^{n+1}(x,y) \Delta t\big)v^{n+1}(x,y)
	\end{gather}
	
	\begin{gather}
		u(x,y)v(x,y) = u_0(x,y)v(x,y)
	\end{gather}
	
	We do the initial condition for descritization purposes.
	
	Take the integral over the area $\Omega$.
	
	\begin{gather}
	\begin{split}
		 \bigintssss_{\Omega} \big( u^{n+1}(x,y) - \nabla^2u^{n+1}(x,y)&\Delta t \big)v^{n+1}(x,y)\,dA  =\\
		& \bigintssss_{\Omega} \big(u^n(x,y) + f^{n+1}(x,y) \Delta t\big)v^{n+1}(x,y)\,dA
	\end{split}
	\end{gather}
	
	\begin{gather}
	\begin{split}
		\bigintssss_{\Omega} u(x,y)v(x,y)\,dA = \bigintssss_{\Omega} u_0(x,y)v(x,y)\,dA
	\end{split}
	\end{gather}
	
	Let's handle the main PDE, and let's call $u=u^{n+1}(x,y)$ which is supposed to be sought at every increment and all state variable of $n+1$.
	
	\begin{gather}
	\begin{split}
	\bigintssss_{\Omega} uv\,dA  - \bigintssss_{\Omega} v\nabla^2u\Delta t \,dA =  \bigintssss_{\Omega} \big(u^n + f \Delta t\big)v\,dA
	\end{split}
	\end{gather}
	
	The second term
	
	\begin{gather}
	\begin{split}
	\Delta t \bigintssss_{\Omega} v\nabla^2u \,dA = \Delta t \bigintssss_{\Omega} \nabla.(v\nabla u) \,dA - \Delta t \bigintssss_{\Omega} \nabla v.\nabla u \,dA 
	\end{split}
	\end{gather}
	
	The second integral can be taken over the boundary.
	
	\begin{gather}
	\begin{split}
	 \bigintssss_{\Omega} \nabla.(v\nabla u) \,dA = \bigintssss_{\partial\Omega} v(\nabla u).\textbf{n} \,dc
	\end{split}
	\end{gather}
	
	so because $v(x,y)$ vanishes on the boundary
	
	\begin{gather}
	\begin{split}
	\bigintssss_{\Omega} \nabla.(v\nabla u) \,dA = \bigintssss_{\partial\Omega} v(\nabla u).\textbf{n} \,dc = 0
	\end{split}
	\end{gather}
	
	and thus the weak form of the PDE becomes
	
	\begin{gather}
	\begin{split}
	\Delta t \bigintssss_{\Omega} v\nabla^2u \,dA = - \Delta t \bigintssss_{\Omega} \nabla v.\nabla u \,dA 
	\end{split}
	\end{gather}
	
	
	\begin{gather}
	\begin{split}
	\bigintssss_{\Omega} uv\,dA + \Delta t \bigintssss_{\Omega} \nabla v.\nabla u \,dA  =  \bigintssss_{\Omega} \big(u^n + f \Delta t\big)v\,dA
	\end{split}
	\end{gather}
	
	\begin{gather}
	\begin{split}
	a_{n+1}(u,v) = \bigintssss_{\Omega} uv\,dA + \Delta t \bigintssss_{\Omega} \nabla v.\nabla u \,dA \\
	L_{n+1}(u,v) = \bigintssss_{\Omega} \big(u^n + f \Delta t\big)v\,dA
	\end{split}
	\end{gather}
	
	and
	
	\begin{gather}
	\begin{split}
	a_{n+1}(u,v) = L_{n+1}(u,v)
	\end{split}
	\end{gather}
	
	
	Second the initial condition.
	
	\begin{gather}
	\begin{split}
	\bigintssss_{\Omega} u(x,y)v(x,y)\,dA = \bigintssss_{\Omega} u_0(x,y)v(x,y)\,dA
	\end{split}
	\end{gather}
	
	and
	
	\begin{gather}
	\begin{split}
	\bigintssss_{\Omega} uv\,dA = \bigintssss_{\Omega} u_0v\,dA
	\end{split}
	\end{gather}
	
	
	\begin{gather}
	\begin{split}
		a_0(u(x,y),v(x,y)) = L_0(u_0(x,y),v(x,y))
	\end{split}
	\end{gather}
	
	We project the values of the initial conditions to the nodal points.
	
	\section*{Solution}
	
	So we are supposed to find the solution
	
	\begin{gather}
	\begin{split}
	u(x,y;t) \in V 
	\end{split}
	\end{gather}
	
	and the test function is
	
	\begin{gather}
	\begin{split}
	v(x,y;t) \in \hat{V}
	\end{split}
	\end{gather}
	
	We do the descritization so at each time $t$
	
	
	\begin{gather}
	\begin{split}
		u = \sum_{i=1}^{N}U_i\phi_i
	\end{split}
	\end{gather}
	
	\begin{gather}
	\begin{split}
		v = \sum_{i=1}^{N}V_i\phi_i
	\end{split}
	\end{gather}
	
	at the time $t=0$
	
	\begin{gather}
	\begin{split}
	u^0 = \sum_{i=1}^{N}U^0_i\phi_i
	\end{split}
	\end{gather}
	
	
	
	\subsection*{Exact Solution}
	
	Let's say the solution is
	
	\begin{gather}
	\begin{split}
	u = 1 + x^2 + \alpha y^2 + \beta t
	\end{split}
	\end{gather}
	
	Pluging it into the PDE gives
	
	\begin{gather}
	\begin{split}
	 \beta = 2 + 2\alpha + f(x,y,t)
	\end{split}
	\end{gather}
	
	thus
	
	\begin{gather}
	\begin{split}
		f(x,y,t) = \beta - 2 - 2\alpha
	\end{split}
	\end{gather}
	
	Let's assume the BC is just what we have for PDE, so
	
	\begin{gather}
	\begin{split}
		u_\mathrm{D} = 1+x^2+\alpha y^2 + \beta t \hspace{5mm} \text{on} \hspace{5mm} (x,y)\in\Omega
	\end{split}
	\end{gather}
	
	so the initial condition is
	
	\begin{gather}
	\begin{split}
	u(x,y,t=0) = u^0 = 1+x^2+\alpha y^2
	\end{split}
	\end{gather}
	
	Thus, the exact solution of the heat equation 
	
	\begin{gather}
	\begin{split}
	\PDfirst{u}{t} = \nabla^2u + \beta - 2 - 2\alpha
	\end{split}
	\end{gather}
	
	with the bounday conditions everywhere on the boundary:
	
	\begin{gather}
	\begin{split}
		u_\mathrm{D} = 1 + x^2+\alpha y^2 + \beta t
	\end{split}
	\end{gather}
	
	and the initial condition everywhere
	
	\begin{gather}
	\begin{split}
		u^0= 1 + x^2+\alpha y^2 
	\end{split}
	\end{gather}
	
	is
	
	\begin{gather}
	\begin{split}
		u(x,y,t) =  1 + x^2+\alpha y^2 + \beta t
	\end{split}
	\end{gather}
	
	
	
	\subsection*{Finite Element Solution}
	
	
	
	
	

\end{document}